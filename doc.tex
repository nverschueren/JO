\documentclass{article}
\usepackage[margin=2cm]{geometry}
\usepackage{amsmath}
\begin{document}
\begin{center}
  {\Large \textbf{Jumping Oscillons SandBox}}
\end{center}
\begin{flushright}
  \begin{tabular}{cc}
    \textbf{Name:}&Adrian van Kan\\
    &Nicolas V. van Rees
  \end{tabular}\\
  \tiny{\today}
  \end{flushright}
  \hrule
\section{Model}
\begin{align}
  \label{eq:mod}
\partial_t u&=d_u\partial_{xx} u +k_1+2u-u^3-k_3v-k_4w\\
\partial_tv&=d_v\partial_{xx} v+\frac{(u-v)}{\tau} \nonumber\\
\partial_tw&=d_w\partial_{xx}w+u-w.\nonumber
\end{align}
\section{How to make this code work in your machine}
Once the dependencies have been satisfied. Running \verb!cmake .!
should allow the user to compile the program by executing the
command \verb!make! in the local directory. if the binary is created
successfully, it can be executed via \verb!./jos! . The program can be
called with two additional arguments, namely the initial condition and
parameter values as:\\
\verb! ./jos solution.dat parameters.dat!\\
More on this below
\section{Usage}
Once the program is running, the user is prompt with an interactive
window called \verb!Instant solution!, showing the space on the
horizontal direction and one of the components of (\ref{eq:mod}) $u$ (blue), $v$ (red),
$w$(green).  The interaction with the user is done through the mouse
and keyboard.



\end{document}
%%% Local Variables:
%%% mode: latex
%%% TeX-master: t
%%% End:
